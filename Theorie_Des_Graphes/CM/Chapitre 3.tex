\documentclass{article}
\usepackage[utf8]{inputenc}
\usepackage[T1]{fontenc}
\usepackage{textcomp}
\usepackage{amsmath,amssymb}
\usepackage{lmodern}
\usepackage[a4paper]{geometry}
\usepackage{graphicx}
\usepackage{xcolor}
\usepackage{microtype}
\usepackage{hyperref}
\usepackage{diagbox}
\usepackage{booktabs}
\usepackage{listings}
\usepackage[francais]{babel}
\usepackage{algorithm}
\usepackage{algorithmic}
\definecolor{darkWhite}{rgb}{0.94,0.94,0.94}
\lstset{
  aboveskip=3mm,
  belowskip=-2mm,
  backgroundcolor=\color{darkWhite},
  basicstyle=\footnotesize,
  breakatwhitespace=false,
  breaklines=true,
  captionpos=b,
  commentstyle=\color{red},
  deletekeywords={...},
  escapeinside={\%*}{*)},
  extendedchars=true,
  framexleftmargin=16pt,
  framextopmargin=3pt,
  framexbottommargin=6pt,
  frame=tb,
  keepspaces=true,
  keywordstyle=\color{blue},
  language=C, JavaScript
  literate=
  {²}{{\textsuperscript{2}}}1
  {⁴}{{\textsuperscript{4}}}1
  {⁶}{{\textsuperscript{6}}}1
  {⁸}{{\textsuperscript{8}}}1
  {€}{{\euro{}}}1
  {é}{{\'e}}1
  {è}{{\`{e}}}1
  {ê}{{\^{e}}}1
  {ë}{{\¨{e}}}1
  {É}{{\'{E}}}1
  {Ê}{{\^{E}}}1
  {û}{{\^{u}}}1
  {ù}{{\`{u}}}1
  {â}{{\^{a}}}1
  {à}{{\`{a}}}1
  {á}{{\'{a}}}1
  {ã}{{\~{a}}}1
  {Á}{{\'{A}}}1
  {Â}{{\^{A}}}1
  {Ã}{{\~{A}}}1
  {ç}{{\c{c}}}1
  {Ç}{{\c{C}}}1
  {õ}{{\~{o}}}1
  {ó}{{\'{o}}}1
  {ô}{{\^{o}}}1
  {Õ}{{\~{O}}}1
  {Ó}{{\'{O}}}1
  {Ô}{{\^{O}}}1
  {î}{{\^{i}}}1
  {Î}{{\^{I}}}1
  {í}{{\'{i}}}1
  {Í}{{\~{Í}}}1,
  morekeywords={*,...},
  numbers=left,
  numbersep=10pt,
  numberstyle=\tiny\color{black},
  rulecolor=\color{black},Usine logicielle javascript
  showspaces=false,
  showstringspaces=false,
  showtabs=false,
  stepnumber=1,
  stringstyle=\color{gray},
  tabsize=4,
  title=\lstname,
}
\hypersetup{pdfstartview=XYZ}
\title{Théorie des graphes - Chapitre 3 : Parcours de graphes }
\author{}
\date{}
\begin{document}
\maketitle{}
\tableofcontents
\newpage
Un parcours va partir d’un sommet. \\
-Il va calculer un ordre de découverte/parcours \\
-Arbre de parcours/Arbre d’exploration \\
\section{Parcours générique}
\begin{algorithm}
\caption{PG($G$,$u$): $GO$}
\begin{algorithmic}
\REQUIRE $G$ un graphe; $u$ un sommet 
\ENSURE $GO$ un graphe orienté avec tous les points marqué.
\STATE // var 
\STATE $L$ : Une liste 
\STATE $x,y$ : Des sommets 
\STATE // init
\STATE NouvelleListe($L$) // L est vide 
\STATE Marquer($u$) // Marquer serait une fonction permettant de dire qu'on est passé par ce sommet.
\STATE $L \leftarrow u$ // On ajoute u à la liste L 
\STATE // process
\WHILE{$\neg$ EstVide(L)} 
\STATE $x \leftarrow $TeteList($L$) // On prend la tête de la liste
\STATE $GO \leftarrow $ExtrairePremier($L$) // On retire la première valeur de la liste \\
\textbf{foreach} $y \in$ Voisinage($x$,$g$) \textbf{do} \\
\hspace{0.5cm} \textbf{if} nonMarquer($y$) \\
\hspace{1cm} Marquer($y$) \\
\hspace{1cm} $L \leftarrow y$ \\
\hspace{0.5cm} \textbf{end if} \\
\textbf{end foreach}
\ENDWHILE 
\RETURN $GO$

\end{algorithmic}
\end{algorithm}
\url{http://www-verimag.imag.fr/~wack/ALGO5/Cours10.pdf} \\
L’algorithme générique permet de déterminer si un graphe est connexe. \\
\\
L’algorithme maintient 3 catégories de sommets \\
1) Les sommets traités \\
2) Les voisins des sommets traités \\
3) Les sommets non traités \\ 

\newpage
\section{Parcours en largeur (Breadth First Search)}
\subsection{L'algorithme}
\begin{algorithm}
\caption{BFS($G$,$u$):}
\begin{algorithmic}
\REQUIRE $G$ un graphe; $u$ un sommet. 
\STATE // var
\STATE $F$ : Une file 
\STATE // init 
\STATE NouvelleFile($F$)
\STATE $F$.enfiler($u$)
\STATE Marquer($u$)
\STATE // process
\WHILE{$\neg $estVide($F$)}
\STATE $u \leftarrow f$.defiler()
\PRINT u
\FORALL{$t \leftarrow $Voisin($u$) $\in G$}
\IF{$\neg$Marquer($S$)}
\STATE $F$.enfiler($t$)
\STATE Marquer(t)
\ENDIF
\ENDFOR
\ENDWHILE
\end{algorithmic}
\end{algorithm}
\url{https://en.wikipedia.org/wiki/Breadth-first_search} \\
Application : Déterminer un plus cours chemin dans un graphe univalué. Déterminer facilement si un graphe est Biparti.  
\subsection{Complexité}
O(n+m) \\
Chaque sommet va être extrait une seule fois de L et ça va déclencher son traitement. (tous ce qui à l’intérieur du while). \\
- extraire de la liste et supprimer O(1) \\
- foreach va être exécutée O(d(v)) fois \\
\\
La complexité va être dominée par $\sum d(v)=2m$
\subsection{Propriété}
On va noter $\delta(u,v)$ la longueur du plus court chemin.
\\
\underline{Lemme 1 :} \\
Soit $G=(V,E)$ un graphe et $s$ un sommet pour toute arête $u$, $v \in E(G)$ on a l’inégalité : \\
(1) $\delta(s,v) \leq \delta(s,u)+1$ \\
\\
\underline{Lemme 2 :} \\
Soit $G(V,E)$ et son sommet de départ on a l’inégalité : \\
 (2) $\delta(s,v) \leq l[v] \forall v \in V(G) \neq s$ \\
 \\ 
\underline{Lemme 3 :} \\
A tout instant t, la liste est composé des sommets $(v_1,v_2,… ,v_k)$ : \\
- $l[vk] \leq l[v_1]+1$ \\
- $l[v_i+1] \forall i \in {1,..,k-1}$ \\
\\
\underline{Théorème : } \\
Soit $G=(V,E)$ un graphne non orienté connexe de sommet $s$ : \\
A la fin de l’algo $l[v]=\delta(s,u)$. \\
\section{Parcours en profondeur(Depth-first search)}
\subsection{L'algorithme}
\subsubsection{Implémentation récursive}
\begin{algorithm}
\caption{Explorer}
\begin{algorithmic}
\REQUIRE $G$ un graphe; $u$ un sommet.
\STATE Marquer($s$)
\PRINT $s$
\FORALL {$t$ voisin de $s$}
\IF{$\neg$Marquer($s$)}
\STATE{Explorer($G,t$)
\ENDIF
\ENDFOR
\end{algorithmic}
\end{algorithm}   
\begin{algorithm}
\caption{DFS\_recursive($G$,$u$):}
\begin{algorithmic}
\REQUIRE $G$ un graphe; $u$ un sommet. 
\FORALL{$u \in G$}
\IF {$\neg$Marquer($S$)}
\STATE Explorer($G,s$)
\ENDIF
\ENDFOR
\end{algorithmic}
\end{algorithm}
\url{https://en.wikipedia.org/wiki/Depth-first_search}
\newpage
\subsubsection{Implémentation classique}
\begin{algorithm}
\caption{DFS($G$,$u$):}
\begin{algorithmic}
\REQUIRE $G$ un graphe; $u$ un sommet. 
\STATE // var
\STATE $P$ : Une pile
\STATE $u$, $v$ : Deux sommets 
\STATE // init 
\STATE NouvellePile($P$)
\STATE $P$.empiler($P$)
\STATE // process
\WHILE{$\neg$ ($P$.estVide())}
\STATE $s \leftarrow p$.depiler()
\IF {$\neg$ Marque($u$)}
\STATE Marquer(S)
\FORALL{Arrêtes de $u$ à $v$ $\in G.$Adjacent($u$)}
\STATE $P$.enfiler($u$)
\ENDFOR
\ENDIF
\ENDWHILE
\end{algorithmic}
\end{algorithm}
\url{https://en.wikipedia.org/wiki/Depth-first_search}
\subsection{Complexité}
Chaque sommet v est traité exactement une fois. \\
Le traitement de chaque sommet $o(d(v)$ donc $o(n+m)$
\subsection{Propriété}
\underline{Théorème : }
 Dans un parcours en profondeur d’un graphe (oriente ou pas) pour toutes paire de sommets $u$ et $v$ les intervalles $I_u$ et $I_v$ ($I_x=[d[x],f[x]]$) \\
 Satisfont exactement une des conditions suivantes : \\
1) $I_u$ et $I_v$ sont disjoint \\
2) $I_u$ est contenu dans $I_v$ \\
3) $I_v$ est contenu dans $I_u$ \\
 \\
 \underline {Classement des arcs : } \\
1) Arcs de liaisons : Relation de parenté \\
2) Arcs arrières : (u,v) si u est un descendant de v \\
3) Arcs avant : (u,v) v est un descendant de u \\
4) Arcs transverse : Tous les autres \\
\\
\underline{ Théorème du chemin blanc : } \\
Dans une arborescence de parcours en profondeur un sommet $u$ est descendant de $v$ si ai moment où v est découvert (à l’instant $d[v]$) il existe dans un chemin uniquement composé de sommets de $V$ à $u$. \\

 

\end{document} 
