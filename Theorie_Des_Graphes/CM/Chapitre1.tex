\documentclass{article}
\usepackage[utf8]{inputenc}
\usepackage[T1]{fontenc}
\usepackage{textcomp}
\usepackage{amsmath,amssymb}
\usepackage{lmodern}
\usepackage[a4paper]{geometry}
\usepackage{graphicx}
\usepackage{xcolor}
\usepackage{microtype}
\usepackage{hyperref}
\usepackage{diagbox}
\usepackage{booktabs}
\usepackage{listings}
\usepackage[francais]{babel}
\definecolor{darkWhite}{rgb}{0.94,0.94,0.94}
\lstset{
  aboveskip=3mm,
  belowskip=-2mm,
  backgroundcolor=\color{darkWhite},
  basicstyle=\footnotesize,
  breakatwhitespace=false,
  breaklines=true,
  captionpos=b,
  commentstyle=\color{red},
  deletekeywords={...},
  escapeinside={\%*}{*)},
  extendedchars=true,
  framexleftmargin=16pt,
  framextopmargin=3pt,
  framexbottommargin=6pt,
  frame=tb,
  keepspaces=true,
  keywordstyle=\color{blue},
  language=C, JavaScript
  literate=
  {²}{{\textsuperscript{2}}}1
  {⁴}{{\textsuperscript{4}}}1
  {⁶}{{\textsuperscript{6}}}1
  {⁸}{{\textsuperscript{8}}}1
  {€}{{\euro{}}}1
  {é}{{\'e}}1
  {è}{{\`{e}}}1
  {ê}{{\^{e}}}1
  {ë}{{\¨{e}}}1
  {É}{{\'{E}}}1
  {Ê}{{\^{E}}}1
  {û}{{\^{u}}}1
  {ù}{{\`{u}}}1
  {â}{{\^{a}}}1
  {à}{{\`{a}}}1
  {á}{{\'{a}}}1
  {ã}{{\~{a}}}1
  {Á}{{\'{A}}}1
  {Â}{{\^{A}}}1
  {Ã}{{\~{A}}}1
  {ç}{{\c{c}}}1
  {Ç}{{\c{C}}}1
  {õ}{{\~{o}}}1
  {ó}{{\'{o}}}1
  {ô}{{\^{o}}}1
  {Õ}{{\~{O}}}1
  {Ó}{{\'{O}}}1
  {Ô}{{\^{O}}}1
  {î}{{\^{i}}}1
  {Î}{{\^{I}}}1
  {í}{{\'{i}}}1
  {Í}{{\~{Í}}}1,
  morekeywords={*,...},
  numbers=left,
  numbersep=10pt,
  numberstyle=\tiny\color{black},
  rulecolor=\color{black},Usine logicielle javascript
  showspaces=false,
  showstringspaces=false,
  showtabs=false,
  stepnumber=1,
  stringstyle=\color{gray},
  tabsize=4,
  title=\lstname,
}
\hypersetup{pdfstartview=XYZ}
\title{Théorie des graphes - Chapitre 1 : Les graphes }
\author{}
\date{}
\begin{document}
\maketitle{}
\tableofcontents
\newpage
\section*{{\underline{Modalité d'examen}}}
L'UE de théorie des graphes est à 6 ECTS (Système européen de transfert et d'accumulation de crédits. \\
L'évaluation se fera en contrôle continue. \\
Ecrit n°1 (Début novembre) -> 33\% \\
Ecrit n°2 (Fin décembre)-> 33\% \\
Projet (TP) -> 33\% \\

 

\section{Introduction}
Un graphe (fini) est défini sur un ensemble de sommet et par un ensemble d’arêtes tel que $ E \subseteq V \times V $ \\ \\
Exemple 1 : \\
$ G=(V,E)$ \\
$ V=\{1,2,3,4\} $ \\
$ E=\{(1,2),(2,3),(3,4),(4,1),(1,3)\} \cup \{(3,2),(4,4)\}$ \\ 
$ (1,2) \neq (2,1)$ \\ 
\\
\includegraphics{Image/Schéma_1.PNG}
Il s'agit d'un graphe orienté. \\ 
\\
Exemple 2 : \\
$ H=(w,F) $ \\
$ w=\{a,b,c,d\} $ \\
$ F=\{\{a,b\},\{b,c\},\{c,d\}\} $ \\ 
\\
\includegraphics{Image/Schéma_2.PNG}
\newpage
\section{Historique}
1726 : Leonard Eules \\
Problème des 7 ponts Konigsberg \\
Si on part d’un pont particulier sur le bord de la rivière est ce qu’on peut revenir à ce même point en passant exactement une fois sur chacun des points. \\

1850 : Hamilton \\
Peut on trouver un cheminement sur un graphe en passant exactement une fois sur chaque sommet. \\
\section{Application}
- Trouver un plus court chemin d’un point A à un point B.\\

- Minimiser le nombre de fréquence nécessaire pour un réseau de téléphonie mobile. \\
\section{Point de vue}
- Algorithmique \\

- Algébrique \\

- Probabiliste \\

- Combinatoire \\

\section{Définition usuelle}
\subsection{Boucle}
Une arête dont le point de départ est le point d’arrivée. 
\newpage
\subsection{Voisinage d'un somment}
L’ensemble des sommets w tel que $ v.w \in E(G)$ \\
$ N_G(v)=\{w \mid \{v,w\} \in E(G) \} $ \\
Voisinage fermé noté : $ V_{G}[v]=N_{G}(v) \cup \{v\} $ \\
\includegraphics{Image/Schéma_3.PNG} \\
$ N(a)=\{b,c,e\} $ \\
$ N[a]=\{b,c,e,a\} $ \\
Voisinage entrant : $ N^{-}(a)=\{y \in V \mid yx \in E\} $ \\
Voisinage sortant : $ N^{+}(a)=\{y \in V \mid xy \in E\} $ \\ 

\subsection{Degrés d'un sommet}
Le degré de v noté $d_{G}(v)$, est le nombre de voisin de v.  \\
$ d_{G}(v)=\mid v(v) \mid $ \\

\includegraphics{Image/Schéma_3.PNG} \\
$ d_{G}(a)=3 $

\subsection{Degrés d'un graphe}
Le degrés minimum d'un graphe noté $ \delta(G) $, est le nombre minimum de voisins pour un sommet dans le graphe. \\
$ \delta(G) = min \mid d(v) \mid v \in V(G)\} $ \\

Le degrés maximum d'un graphe noté $ \Delta(G) $, est le nombre maximum de voisins pour un sommet dans le graphe. \\
$ \Delta(G) = max \mid d(v) \mid v \in V(G)\} $
\newpage 
\includegraphics{Image/Schéma_3.PNG} \\
$ \delta(G)=2 $ \\
$ \Delta(G)=3 $ \\

\subsection{Chemin}
Un chemin P est une séquence linéaire du sommets  $(v_1,v_2,…,v_{k })$ tous les $ v_i $ sont distincts.
$ {v_i,v_{i+1}\} \in E(G) $ pour tous les $ i \in \{1,…,k-1\} $.

\subsection{Cycle}
Un cycle est une séquence circulaire $ (v_1,v_2,…,v_k,v_1) $ \\
$ v_i,v_i+1(\%k) \in E(G) $

\subsection{Marche}
Une marche (fermé) est une séquence d’arêtes qui sont consécutives sans contrainte de répétions. 

\subsection{Sous-graphe}
Un sous-graphe est un graphe partiel. \\
Soit $ G=(V,E) $ est un graphe, $ H=(W,F) $ est un sous-graphe de $ G $ si $ W  \subseteq V \textsf{et} F\subseteq E $ \\ 

Soit $ G=(V,E) $ est un graphe, $ H=(W,F) $ est un sous-graphe induit de $ G $ si $ W\subseteq F $ et $ F=E \cap (w \times w) $ \\
$ H $ est souvent induit si on peut obtenir $ H $ à partir de $ G $ en supprimant des sommets \\

\subsection{Mineur de graphe}
Soit $ G=(V,E) $ est un graphe, $ H=(W,F) $ si $H$ peut être obtenue à partir de $G$ alors : \\
- Supprimer des sommets isolés \\
- Supprimer des arêtes \\
- Contracter d’arêtes \\

\newpage
\subsection{Complémentaire de graphe}
Soit $G=(V,E)$ un graphe non orienté. Le complémentaire de $G$ noté $\overline{G}$ avec $\overline{G}=(V,\overline{G})$. \\
$\overline{E}=\begin{pmatrix} V \\ 2 \end{pmatrix} \textbackslash 2$
\\
$\overline{G}=(V,\overline{G})$ est l’ensemble des couples $v_i, v_j \forall v_i, v_j  \in V $ et $v_i \neq v_j$ \\

\subsection{Isomorphisme de graphe}
Deux graphes qui sont « égaux ». \\
$G=(V,E)$ et $H=(W,E)$ sont isomorphe si et seulement si il existe une fonction bijective $v \rightarrow w$ tel que u,v $\in E(G)$ si et seulement si $\varphi (u), \varphi (v) \in E(H)$.
\\
G isomorphe à H noté G \sim H.

\section{Fonction de graphes remarquable}
\subsection{Stable ou ensemble indépendant}
Ensemble de sommet qui ne contient qui ne contient pas d’arêtes. \\
$S=(V,E)$ est un ensemble de sommet où $E= \emptyset$

\subsection{Clique complémentaire des stables}
$K=(V,E)$ $E=\begin{pmatrix} V \\ 2 \end{pmatrix}$ toutes les arêtes sont présents dans le graphe une clique sur n sommets notée $k_n$. \\
\includegraphics{Image/Schéma_4.PNG} \\

\subsection{Forêts}
Un graphe sans cycle

\subsection{Arbre}
Un graphe sans cycle et connexe

\subsection{Graphe biparti}
Un graphe $G=(v,E)$ pour lequel on peut partitionner. \\
$V=A \cup B (A \cap B=\emptyset)$ \\
tel que G[A] (le graphe induit par A) et G[B] forment des stables.
  
\newpage
\subsection{Graphe connexe}
Un graphe $G’=(V,E)$ est connexe si et seulement si entre chaque paire de sommets $a,b$ il existe un chemin dans $G$ qui relie les sommets $a$ et $b$.

\subsection{Composante connexe}
Une composante connexe de graphe $G=(v,E)$ est un sous ensemble maximal S de sommets de G tel que G[S] est connexe. \\
Remarque : L’ensemble des composantes connexes d’un graphe forment une partition des sommets. Cette partition est unique.

\section{La représentation des graphes en machine}
\subsection{La matrice adjacence}
Soit M avec n lignes et n colonnes. \\
Chaque lignes et chaque colonnes représentent un sommet. \\
$ M_{i,j}=\left\{\begin{array}{rl}

	1 & \mbox{si } (v_i,v_j)\in E \\

        0 & \mbox{sinon.}

\end{array}\right. $
\\
Exemple : \\
\includegraphics{Image/Schéma_5.png}
\\
$M=\begin{pmatrix}
1 & 1 & 0 & 0 & 1 & 0\\
1 & 0 & 1 & 0 & 1 & 0\\
0 & 1 & 0 & 1 & 0 & 0\\
0 & 0 & 1 & 0 & 1 & 1\\
1 & 1 & 0 & 1 & 0 & 0\\
0 & 0 & 0 & 1 & 0 & 0
\end{pmatrix} $

\newpage
Soit G un graphe avec n sommets et m arrêtes. \\
- espace : $O(n^2)$ \\
- Tester l’adjacence de deux sommets : $O(1)$ \\
- Connaître les voisins : $O(n)$ \\
\\

\subsection{La matrice d'incidence}
Les lignes représentent les sommets et les colonnes représentent les arêtes. 
le coefficient de la matrice d'incidence en ligne $i$ et en colonne $j$ vaut :\\
\\
1 si le sommet $v_i$ est une extrémité de l'arête $x_j$ \\
2 si l'arête $x_j$ est une boucle sur $v_i$ \\
0 sinon \\
Exemple :
\\
\includegraphics{Image/Schéma_6.png}
\\
Prenons le cas du graphe ci-contre. Il possède 5 sommets et 6 arêtes, la matrice d'incidence aura donc 5 lignes et 6 colonnes : \\
le sommet 1 est l'aboutissement des arêtes 1 et 5 \\
le sommet 2 est l'aboutissement des arêtes 1, 2 et 6 \\
le sommet 3 est l'aboutissement des arêtes 2 et 3 \\
le sommet 4 est l'aboutissement des arêtes 3 et 4 \\
le sommet 5 est l'aboutissement des arêtes 4, 5 et 6 \\
\\
$M=\begin{pmatrix}
1 & 0 & 0 & 0 & 1 & 0\\
1 & 1 & 0 & 0 & 0 & 1\\
0 & 1 & 1 & 0 & 0 & 0\\
0 & 0 & 1 & 1 & 0 & 0\\
0 & 0 & 0 & 1 & 1 & 1\\
\end{pmatrix}$ \\

\newline
\subsection{Liste adjacente} 
Une première liste qui correspond au sommet. \\ 
Pour chaque sommet on la liste des voisins. \\ 
Exemple : \\
\includegraphics{Image/Schéma_7.png}
\\
Soit G un graphe avec n sommets et m arrêtes. \\
- Espace : $O(n+m)$\\
- Tester l’adjacence de deux sommets : $O(min \{ d(u), d(v) \})$ \\
- Connaître les voisins : $O( d(v) )$ \\

\end{document} 
