\documentclass{article}
\usepackage[utf8]{inputenc}
\usepackage[T1]{fontenc}
\usepackage{textcomp}
\usepackage{amsmath,amssymb}
\usepackage{lmodern}
\usepackage[a4paper]{geometry}
\usepackage{graphicx}
\usepackage{xcolor}
\usepackage{microtype}
\usepackage{hyperref}
\usepackage{diagbox}
\usepackage{booktabs}
\usepackage{listings}
\usepackage[francais]{babel}
\usepackage{algorithm}
\usepackage{algorithmic}
\definecolor{darkWhite}{rgb}{0.94,0.94,0.94}
\lstset{
  aboveskip=3mm,
  belowskip=-2mm,
  backgroundcolor=\color{darkWhite},
  basicstyle=\footnotesize,
  breakatwhitespace=false,
  breaklines=true,
  captionpos=b,
  commentstyle=\color{red},
  deletekeywords={...},
  escapeinside={\%*}{*)},
  extendedchars=true,
  framexleftmargin=16pt,
  framextopmargin=3pt,
  framexbottommargin=6pt,
  frame=tb,
  keepspaces=true,
  keywordstyle=\color{blue},
  language=C, JavaScript
  literate=
  {²}{{\textsuperscript{2}}}1
  {⁴}{{\textsuperscript{4}}}1
  {⁶}{{\textsuperscript{6}}}1
  {⁸}{{\textsuperscript{8}}}1
  {€}{{\euro{}}}1
  {é}{{\'e}}1
  {è}{{\`{e}}}1
  {ê}{{\^{e}}}1
  {ë}{{\¨{e}}}1
  {É}{{\'{E}}}1
  {Ê}{{\^{E}}}1
  {û}{{\^{u}}}1
  {ù}{{\`{u}}}1
  {â}{{\^{a}}}1
  {à}{{\`{a}}}1
  {á}{{\'{a}}}1
  {ã}{{\~{a}}}1
  {Á}{{\'{A}}}1
  {Â}{{\^{A}}}1
  {Ã}{{\~{A}}}1
  {ç}{{\c{c}}}1
  {Ç}{{\c{C}}}1
  {õ}{{\~{o}}}1
  {ó}{{\'{o}}}1
  {ô}{{\^{o}}}1
  {Õ}{{\~{O}}}1
  {Ó}{{\'{O}}}1
  {Ô}{{\^{O}}}1
  {î}{{\^{i}}}1
  {Î}{{\^{I}}}1
  {í}{{\'{i}}}1
  {Í}{{\~{Í}}}1,
  morekeywords={*,...},
  numbers=left,
  numbersep=10pt,
  numberstyle=\tiny\color{black},
  rulecolor=\color{black},Usine logicielle javascript
  showspaces=false,
  showstringspaces=false,
  showtabs=false,
  stepnumber=1,
  stringstyle=\color{gray},
  tabsize=4,
  title=\lstname,
}
\hypersetup{pdfstartview=XYZ}
\title{Théorie des graphes - Chapitre 2 : Décomposition en cycles et tour Euleriens }
\author{}
\date{}
\begin{document}
\maketitle{}
\tableofcontents
\newpage
\section{Caractérisation des graphes bipartis}
\subsection*{Théorème}
Un graphe est bipartis si et seulement si il ne contient pas de cycle impair.
\subsection*{Démonstration}
\subsubsection*{$\Rightarrow$}
Si le graphe a un cycle impair alors il n’est pas biparti.
\subsubsection*{$\Leftarrow$}
Si le graphe n’a pas de cycle impairs tous les cycles du graphes sont pairs. \\
Si le graphe est biparti toutes les composantes connexe du graphe sont biparti donc on peut analyser  sur les cas connexes. (1) \\
Soit un graphe $G=(V,E)$ connexe par (1). \\
Prenons un sous arbre couvrant $T=(V,F)$ de $G$. \\
On l’enracine en un sommet quelconque. \\
On peut définir un nombre partiel sur les sommets du graphe. \\
deux sommets x et y sont comparables sans T si x est sur le chemin de y à T. \\  
Partition des sommets qui est obtenu à partir d’un arbre couvrant et une racine arbitraire r. \\
A sommets distant paire de r dans T. \\
B sommets distant impaire de r dans T. \\
Pour toute arête x et y de G. \\
x et y sont dans deux partie différentes. \\
\subsection*{Étude de cas}
\subsubsection*{1- $x,y \in E(T)$}
Par construction sans perte de généralisé y est à distance paire de r et comme x est relié à y x est à distance impaire de r donc x et y sont dans deux parties différentes.
\subsubsection*{2- Si $x,y \in E(G) \backslash E(I)$}
$xy + T$ contient un cycle qui passe par x y. \\
Si x et y sont dans le même ensemble, la parité su chemin qui relie x à y dans l’arbre sera paire  forme un cycle impaire. \\
x et y sont dans deux ensembles différents. 
\newpage
\section{Décomposition en cycles}
La décomposition F d’un graphe $G=(V,E)$ est une partition des arêtes de G. $ F=\{ F_1, F_2,…, F_k \}$. \\
$(F_1 \subseteq E(G); F_i \cap F_j = \emptyset$ $\forall_{i,j} \in \{1,...,k\}, i \neq j\} \cup F_i=E(G))$. \\
Tous les membres $F_i$ de F respectent une condition (2).
(2) est un chemin dont chaque arête forment eux mêmes un chemin. \\
Tous graphe admet une décomposition en chemin. \\
\subsection*{Lemme}
Soit $G=(V,E)$ un graphe non orienté, si $ \delta(g) \geq 2$ alors G contient au moins un cycle.
\subsection*{Preuve}
Si G contient une boucle c’est vrai, si G ne contient pas de boucle on prend un chemin $P=v_1,v_2,…,v_k$ de longueur maximum. \\
$v_{k+1}$ n’est pas un autre sommet que les sommets du chemin si non le chemin n’est pas maximum. \\
$v'$ est un sommet dans $\{v_{1},v_{2},…,v_{k-2}\}$. \\
$v=v_{j}$. \\
$(v_k,v_j,v_{j+1},…,v_{k-1},v_k)$ forme un cycle. \\
\\
\subsection*{Définition}
Un graphe $G=(V,E)$ est pair si et seulement si pour chaque sommet v de G on a $d(v)$ pair. \\
Si G admet une décomposition en cycle alors G est pair.
\subsection*{Preuve}
Pour chaque cycle $C_i$ de décomposition, quel que soit le sommet v de G. \\
$d(C_i)=2$ si $ v \in C_i$, $0$ si $ v \notin C_i$. \\
\newpage
\subsection*{Théorème de Veblen}
G admet une décomposition en cycle si et seulement si G est pair.
\subsection*{Preuve} 
\subsubsection*{$\Rightarrow$ }
Prouvé dans la définition précédente
\subsubsection*{$\Leftarrow$}
Par induction descendent sur le nombre de sommets, \\
Notre graphe, tous les sommets sont de dégrée strictement positif. \\
Si G est pair à degré positif $\delta(G) \geq 2$ par la lemme précédente G possède un cycle C. \\
$G’=G \setminus C$ \\
G’ est un graphe pair car tous les sommets v de c on $d_{G’}(v)=d_{G}(v)-2$. \\
On considère G’’=G’-tous les sommets de degré 0. \\
On peut ré appliquer la procédure sur G’’.  \\
\section{Graphe Eulérien}
\subsection*{Définition}
Un graphe est eulérien si et seulement si il admet une marche fermée qui passe exactement une fois sur chaque arêtes.\\
\\
\subsection*{Notation}
 Soit x et y deux sous-ensembles de sommets(pas forcément disjoint) on note E[x,y] l’ensemble des arêtes qui ont une extrémités dans x et l’autre dans y. Le nombre d’arêtes E[x,y]=(?). \\
Quand y=V \backslash X. \\
$E=[x,y]=\partial(x)=\partial(y)$ \\
\subsection*{Définition}
Une arête e de G (connexe) est dite décontractante si et seulement si G est déconnecté. 
\subsection*{Proposition}
Une arête e est déconnectante si et seulement si elle ne participe à aucun cycle de G.
\newpage
\subsection*{Problématique}
Quels sont les graphes qui admettent un tour Eulerien ?
\subsection*{Condition nécessaire}
Si on a tour eulérien pour tout sommet v le tour va arriver sur v autant de fois qu’il part de v donc le degré de v est pair. 
\subsection{Algorithme de Fleury}
\begin{algorithm}
\caption{Fleury($G$,$u$): $w$}
\begin{algorithmic}
\REQUIRE $G$ un graphe connexe et pair; u un sommet de $G$. 
\ENSURE $C$ un tour eulérien qui commence à $u$ et termine à $u$.
\STATE // Var
\STATE F : Un sous graphe de $G$. 
\STATE x : Sommet de $G$.
\STATE C : Chaîne de $G$.
\STATE // Init
\STATE $C \leftarrow u$
\STATE $x \leftarrow u$
\STATE $F \leftarrow G$
\STATE // process
\WHILE {$\delta_{F}(x) \neq 0$}
\STATE Choisir une arêtes $e=xy$ tel que e n'est pas un isthme de $F$ sauf si c'est la seule possibilité.
\STATE $C \leftarrow Cey$
\STATE $x \leftarrow y$
\STATE $F \leftarrow F \setminus E$
\ENDWHILE
\RETURN $C$
\end{algorithmic}
\end{algorithm}
\emph{isthme\footnote{Déconnéctante}}
\subsubsection{Complexité}
Voir TD2 
\subsubsection{Théorème}
Si G est pair alors l’algorithme de Fleury calcule un tour Eulerien de $G$.
 
\end{document} 
