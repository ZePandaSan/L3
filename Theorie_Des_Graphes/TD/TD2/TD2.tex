\documentclass{article}
\usepackage[utf8]{inputenc}
\usepackage[T1]{fontenc}
\usepackage{textcomp}
\usepackage{amsmath,amssymb}
\usepackage{lmodern}
\usepackage[a4paper]{geometry}
\usepackage{graphicx}
\usepackage{xcolor}
\usepackage{microtype}
\usepackage{hyperref}
\usepackage{diagbox}
\usepackage{booktabs}
\usepackage{listings}
\usepackage[francais]{babel}
\usepackage{algorithm}
\usepackage{algorithmic}
\definecolor{darkWhite}{rgb}{0.94,0.94,0.94}
\lstset{
  aboveskip=3mm,
  belowskip=-2mm,
  backgroundcolor=\color{darkWhite},
  basicstyle=\footnotesize,
  breakatwhitespace=false,
  breaklines=true,
  captionpos=b,
  commentstyle=\color{red},
  deletekeywords={...},
  escapeinside={\%*}{*)},
  extendedchars=true,
  framexleftmargin=16pt,
  framextopmargin=3pt,
  framexbottommargin=6pt,
  frame=tb,
  keepspaces=true,
  keywordstyle=\color{blue},
  language=C, JavaScript
  literate=
  {²}{{\textsuperscript{2}}}1
  {⁴}{{\textsuperscript{4}}}1
  {⁶}{{\textsuperscript{6}}}1
  {⁸}{{\textsuperscript{8}}}1
  {€}{{\euro{}}}1
  {é}{{\'e}}1
  {è}{{\`{e}}}1
  {ê}{{\^{e}}}1
  {ë}{{\¨{e}}}1
  {É}{{\'{E}}}1
  {Ê}{{\^{E}}}1
  {û}{{\^{u}}}1
  {ù}{{\`{u}}}1
  {â}{{\^{a}}}1
  {à}{{\`{a}}}1
  {á}{{\'{a}}}1
  {ã}{{\~{a}}}1
  {Á}{{\'{A}}}1
  {Â}{{\^{A}}}1
  {Ã}{{\~{A}}}1
  {ç}{{\c{c}}}1
  {Ç}{{\c{C}}}1
  {õ}{{\~{o}}}1
  {ó}{{\'{o}}}1
  {ô}{{\^{o}}}1
  {Õ}{{\~{O}}}1
  {Ó}{{\'{O}}}1
  {Ô}{{\^{O}}}1
  {î}{{\^{i}}}1
  {Î}{{\^{I}}}1
  {í}{{\'{i}}}1
  {Í}{{\~{Í}}}1,
  morekeywords={*,...},
  numbers=left,
  numbersep=10pt,
  numberstyle=\tiny\color{black},
  rulecolor=\color{black},Usine logicielle javascript
  showspaces=false,
  showstringspaces=false,
  showtabs=false,
  stepnumber=1,
  stringstyle=\color{gray},
  tabsize=4,
  title=\lstname,
}
\hypersetup{pdfstartview=XYZ}
\title{TD2 Théorie des graphes}
\author{}
\date{}
\begin{document}
\maketitle{}
\section*{Exercice 1}
G=(V,E) \\
Si on supprime une arête soit : \\
- On divise une composante en 2 (arête déconnectante) \\
- Une composante perd une arête mais reste connexe donc le nombre de composante d'un graphe modifié est au maximum de h+1.

\section*{Exercice 2}
\subsection*{(1)}
e est deconnectante si et seulement si elle n'appartient à aucun cycle. Un graphe est pair ssi il admet une décomposition en cycle. \\
Donc chaque arête de G est au moins sur un cycle donc pas deconnectante.
\subsection*{(2)}
Soit V le sommer qu'on supprime et soient $C_1,C_2,...,C_{k}$ les composantes connexes. \\
Le nombre de voisins de v dans chaque composantes connexe et pair. \\
Chaque sommet v participe exactement à $\frac{d(v)}{2}$ cycles de la décomposition. Un cycle qui passe par v reste dans la même composantes connexe. Donc le nombre de voisin dans une composante connexe $C_i$ est égale aux nombre de cycles qui passe par V et qui sont contenus dans $C_i \cup \{v\}$. Le nombre de voisin est pair. 

\section*{Exercice 3}
Chaque sommet v participe exactement à $\frac{d(v)}{2}$ cycles de la décomposition.

\section*{Exercice 4}
Le nombre de cycles dans une décomposition en cycle peut varier. \\
Remarque : Trouver une décomposition avec le nombre minimum de cycles est un problème difficile. \\

\section*{Exercice 5}
Tester si une arête est déconnectante : o(n+m) \\
Donc : o($m^2$)

\section*{Exercice 6}

\section*{Exercice 7}
Il ne s'agit pas d'un parcours en largeur car après $a$ soit on choisit $g$ soit on choisit $b$ donc on  doit prendre $f$ et $c$ seront traité directement.

\section*{Exercice 8}
Soit $l[x] \le l[y]$,
Sur l'arête e $y$ peut être découvert par $x$ ou être découvert en même temps que $x$, le niveau ne peut donc pas être supérieur à 1. 

\section*{Exercice 9}
Algorithme de parcours en largeur avec coloriage. \\ 
Si le graphe est Biparti les sommets au niveau pairs forment une partie et les sommets au niveau impair forment une partie(x) et les sommets au niveau impairs forment l'autre partie. \\
\\
\underline{Proposition} : G Biparti si et seulement si chaque niveau est un stable. \\
\underline{Preuve} : \\ 
$\Rightarrow$ si $G$ Biparti tous les niveaux pair correspondent à $x$ et impairs correspondent à $y$. \\
$\Leftarrow$ si tous les niveaux du parcours sont des stables alors $G$ est biparti. \\
\\
On sait qu'il n'y a aucune arête entre deux niveaux $i$ et $j$ de même parité on a $\mid i-j \mid \ge 2$ pour toutes les arêtes $e=\{x,y\}$ de $G$ dans un BFS on a $l[x]-l[y] \le 1$ et comme chaque niveau induit un stable. On obtient bien une bipartition à la fin. \\
\underline{Proposition} : G biparti si et seulement si $\forall e=\{x,y} \in E(G)$ on a $\mid l[x]-l[y]\mid=1$
\section*{Exercice 10}
Diamètre d'un graphe : Le plus long des plus courts chemins. \\
Diam($G$)=max$\{\delta(x,y) \mid \forall x,y \in V(G); x \neq y\}$ 
\\
\\
Lorsqu'on fait un BFS à partir d'un sommet v. On calcule tous les plus courts chemins de v vers tous les autres sommets. \\
- Distance : Donné par $l[w] \forall w \in V(G)$ \\
- Chemin : de w à v (avec la relation de parenté). 
Dans un graphe à n sommets il y a $\frac{n^2-n}{2}$ pair de sommets différentes.
Un BFS$\rightarrow n-1$ \\
On fait $n$ BFS : Un à partir de chaque sommet. \\
On garde le max des niveaux pour chaque pour chaque BFS. \\
Au final on a examiné n(n-1) pairs (chaque pair deux fois).

\end{document}
