\documentclass{article}
\usepackage[utf8]{inputenc}
\usepackage[T1]{fontenc}
\usepackage{textcomp}
\usepackage{amsmath,amssymb}
\usepackage{lmodern}
\usepackage[a4paper]{geometry}
\usepackage{graphicx}
\usepackage{xcolor}
\usepackage{microtype}
\usepackage{hyperref}
\usepackage{diagbox}
\usepackage{booktabs}
\usepackage{listings}
\usepackage[francais]{babel}
\usepackage{algorithm}
\usepackage{algorithmic}
\definecolor{darkWhite}{rgb}{0.94,0.94,0.94}
\lstset{
  aboveskip=3mm,
  belowskip=-2mm,
  backgroundcolor=\color{darkWhite},
  basicstyle=\footnotesize,
  breakatwhitespace=false,
  breaklines=true,
  captionpos=b,
  commentstyle=\color{red},
  deletekeywords={...},
  escapeinside={\%*}{*)},
  extendedchars=true,
  framexleftmargin=16pt,
  framextopmargin=3pt,
  framexbottommargin=6pt,
  frame=tb,
  keepspaces=true,
  keywordstyle=\color{blue},
  language=C, JavaScript
  literate=
  {²}{{\textsuperscript{2}}}1
  {⁴}{{\textsuperscript{4}}}1
  {⁶}{{\textsuperscript{6}}}1
  {⁸}{{\textsuperscript{8}}}1
  {€}{{\euro{}}}1
  {é}{{\'e}}1
  {è}{{\`{e}}}1
  {ê}{{\^{e}}}1
  {ë}{{\¨{e}}}1
  {É}{{\'{E}}}1
  {Ê}{{\^{E}}}1
  {û}{{\^{u}}}1
  {ù}{{\`{u}}}1
  {â}{{\^{a}}}1
  {à}{{\`{a}}}1
  {á}{{\'{a}}}1
  {ã}{{\~{a}}}1
  {Á}{{\'{A}}}1
  {Â}{{\^{A}}}1
  {Ã}{{\~{A}}}1
  {ç}{{\c{c}}}1
  {Ç}{{\c{C}}}1
  {õ}{{\~{o}}}1
  {ó}{{\'{o}}}1
  {ô}{{\^{o}}}1
  {Õ}{{\~{O}}}1
  {Ó}{{\'{O}}}1
  {Ô}{{\^{O}}}1
  {î}{{\^{i}}}1
  {Î}{{\^{I}}}1
  {í}{{\'{i}}}1
  {Í}{{\~{Í}}}1,
  morekeywords={*,...},
  numbers=left,
  numbersep=10pt,
  numberstyle=\tiny\color{black},
  rulecolor=\color{black},Usine logicielle javascript
  showspaces=false,
  showstringspaces=false,
  showtabs=false,
  stepnumber=1,
  stringstyle=\color{gray},
  tabsize=4,
  title=\lstname,
}
\hypersetup{pdfstartview=XYZ}
\title{TD3 Théorie des graphes}
\author{}
\date{}
\begin{document}
\maketitle{}
\section*{Exercice 1}
\subsection*{Graphe G}
. \\
a : (1,20) \\
b : (8,13) \\
c : (9,12) \\
d : (10,11) \\
e : (6,15) \\
f : (7,14) \\
g : (3,18) \\
h : (4,17) \\
i : (5,16) \\
j : (2,15) \\
\subsection*{Graphe H}
a : (1,12) \\
b : (2,11) \\
c : (18,19) \\
d : (17,20) \\
e : (13,16) \\
f : (5,6) \\
g : (4,7) \\
h : (8,9) \\
i : (14,15) \\
j : (3,10) 

\section*{Exercice 2}
\subsection*{(1)}
Chaque sommet v est traité exactement une fois. \\
Le traitement de chaque sommet $o(d(v)$ donc $o(n+m)$
\newpage
\subsection*{(2)}
. \\
Arc liaison (u,v): \\
u : gris \\
v : blanc \\
d[u] est initialisé à un entier \\
f[u]=$\infty$ \\
v n'a pas de date \\
\\ 
Arc arrières (u,v): \\
u: gris (début mais pas fin) \\
v: gris \\
$d[v]<d[u]$ \\ 
\\
Arc avant (u,v): \\
u: gris \\
v: noir \\
d[u]<d[v]<f[v]
\\
Arc transverse (u,v): \\
u : gris \\
v : noir \\
f[v]<d[u]  
\end{document}
