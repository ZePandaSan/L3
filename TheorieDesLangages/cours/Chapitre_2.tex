%pdflatex -halt-on-error -aux-directory=tmp -output-directory=tmp rapport.tex%

\documentclass{article}
\usepackage{amsmath}
\usepackage[utf8]{inputenc}
\usepackage[T1]{fontenc}
\usepackage{graphicx}
\usepackage{hyperref}
\usepackage[francais]{babel}
\usepackage{listings}
\usepackage{xcolor}

\definecolor{codegreen}{rgb}{0,0.6,0}
\definecolor{codegray}{rgb}{0.5,0.5,0.5}
\definecolor{codepurple}{rgb}{0.58,0,0.82}
\definecolor{backcolour}{rgb}{0.95,0.95,0.92}

\lstdefinestyle{mystyle}{
    language=python,
    backgroundcolor=\color{backcolour},   
    commentstyle=\color{codegreen},
    keywordstyle=\color{magenta},
    numberstyle=\tiny\color{codegray},
    stringstyle=\color{codepurple},
    basicstyle=\ttfamily\footnotesize,
    breakatwhitespace=false,         
    breaklines=true,                 
    captionpos=b,                    
    keepspaces=true,                 
    numbers=left,                    
    numbersep=5pt,                  
    showspaces=false,                
    showstringspaces=false,
    showtabs=false,                  
    tabsize=2
}

\lstset{style=mystyle}

\title{Théorie des langages II}
\author{Wassim SAIDANE}
\date{01/03/2021}

\begin{document}
    \pagenumbering{gobble}
    \maketitle
    \pagenumbering{arabic}
    \section*{Note : }
    Ce cours est ma prise de note du cours de L3 infos de Théorie des langages II de Mamadou Kante
    \section*{Chapitre 2 : Automates à pile (non terminé)}
    L'objectif de ce chapitre est de montrer que une correspondance entre grammaires algébriqus et automates à pile. \\
    \\
    Un automate à pile est un automate à états finis muni d'une pile. A chaque étape, l'état suivant est détérminé par l'état courant, la lettre $u$ et l'état de la pile. \\
    \\
    \underline{Définition 2.1} Un automate à pile est un tuple $A=(Q, \Sigma, \Gamma, \delta, q_0, F)$ tel que : 
    \begin{enumerate}
        \item $Q$ est un ensemble fini, appelé ensemble des états. 
        \item $\Sigma$ : Est un ensemble fini, alphabet des mots à reconnaitre. 
        \item $\Gamma$ : Ensemble fini, alphabet de la pile. 
        \item $\delta$ : $Q \times (\Sigma \bigcup \{\epsilon\}) \times (\Gamma \bigcup \{\epsilon\}) \rightarrow 2^{Q \times (\Gamma \bigcup \{\epsilon\})}$
        \item $q_0 \in Q : $Etat initial. 
        \item $F \subseteq Q :$ Etats finaux 
    \end{enumerate}
    \underline{Exécutions d'un automate à pile}
    Une exécution accptante d'un mot $w \in \Sigma^R$ par $A$: Si on peut récrire $w$ en $w_1,w_2,w_3,..,w_m$, où $w_i \in \Sigma \bigcup \{\epsilon\}$ (on ajoute des $\sigma$ entre les lettres de $w$) et si on a une séquance d'états $r_0,r_1,...,r_m$ et de mots $s_0,s_1,s_2,..,s_m \in \Gamma^*$ tels que : \\
    \begin{enumerate}
        \item $r_0=q_0$, $s_0=\epsilon$ (on commence par l'état initial et une pile vide)
        \item $\forall 0 \le i \le m-1, (r_{i+1},b) \in \delta (r_i, w_{i+1}, a)$ où $s_i=at, s_{i+1}=bt, a,b \in \Gamma \bigcup \{\epsilon\} t \in \Gamma^*$ \\
        Si on est à l'état $r_i$, on lit la lettre $w_{i+1}$ (qui peut être la lettre $\epsilon$), la tête de la pile c'est $a$(ou $\epsilon$), on passe à l'état $r_{i+1}$, on empile la lettre $b$ dans la pile ($b$ peut être vide, signifiant on n'empile rien). \\
        \item $r_m \in F$
    \end{enumerate}
    Le langage de $A$, noté $L(A)$, l'ensemble des mots $w$ tel que il existe une exécution acceptante de $A$ sur $w$.
\end{document}
